In this thesis, we investigate cooperation by applying OFDM signals to cooperative relay networks.  We consider the single path relay network and the multiple path relay network.  Using the amplify-and-forward relay algorithm, we derive the input-output relations and mutual informations of both networks.  Using a power constraint at each relay, we consider two relay power allocation schemes.  The first is constant gain allocation, where the amplifying gain used in the amplify-and-forward algorithm is constant for all subcarriers.  The second is equal power allocation, where each subcarrier transmits the same power.  The former scheme does not require CSI (channel state information), while the latter one does.  We simulate the mutual informations using the two relay power allocation schemes.  Results indicate that equal power allocation gives a slightly higher mutual information for the single path relay network.  For the multiple path network, the mutual information is practically the same for both schemes.  Using the decode-and-forward relay algorithm, we derive the input-output relations for both networks.  The transmitter and each relay are assumed to have uniform power distributions in this case.  We simulate the BER (bit error rate) and WER (word error rate) performance for the two networks using both the amplify-and-forward and decode-and-forward relay algorithms.  For the single path relay network, amplify-and-forward gives very poor performance, because as we increase the distance between the transmitter and receiver (and thus, add more relays), more noise and channel distortion enter the system.  Decode-and-forward gives significantly better performance because noise and channel distortion are eliminated at each relay.  For the multiple path relay network, decode-and-forward again gives better performance than amplify-and-forward.  However, the performance gains are small compared to the single path relay network case.  Therefore, amplify-and-forward may be a more attractive choice due to its lower complexity.