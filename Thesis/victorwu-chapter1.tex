\chapter{Introduction}
\label{chap:introduction}

Wireless communication systems inherently suffer from multipath propagation and channel fading.  Time diversity, space diversity, frequency diversity \cite{book:Proakis01}, and combinations of the three are traditionally used to combat these effects.  More recently, \emph{relays} situated between the transmitter and receiver are also being exploited to improve information transfer.  The relays are a network of transceiver nodes between the transmitter and receiver that facilitate the transfer of information.  Thus, the relay network as a whole is an equivalent channel between the transmitter and receiver.  This type of scheme is known as \emph{cooperation} or \emph{cooperative communications} in the literature because the relay network is cooperating with the transmitter and receiver to improve performance.  In this thesis, we consider cooperation in the context of orthogonal frequency division multiplexing (OFDM) systems.

\section{Motivation}
The motivation for cooperative communications is obvious.  Cellular phones, laptops and personal digital assistants (PDAs) are just three examples of wireless devices that are very prevalent today.  These transceiver devices usually communicate independently from each other.  As the authors in \cite{article:Laneman01} note, this is wasting the broadcast nature of the wireless medium.  For example, if a base station is communicating with a user's cellular phone, his/her nearby laptop has the capability to receive the base station's signals and relay them to the phone, improving the end-to-end performance of the base station-phone link.  Unfortunately, laptops and cellular phones today are not designed this way.  This illustration is an example of an \emph{ad-hoc network}, where nodes spontaneously recognize each other and cooperate.  In this thesis, we investigate \emph{structured networks}, where each node knows the existence of all the other nodes a priori.  Whether the nodes discover each other through an ad-hoc algorithm or they are pre-programmed to have this knowledge is beyond the scope of this thesis.  

\section{Related Literature}
The authors in  \cite{article:Sendonaris01}, \cite{article:Sendonaris02} have considered cooperation between intra-cell users in a code division multiple access (CDMA) cellular network.  In this case, cooperation results in higher data rates and leads to lower power requirements for users.  As well, the system is less sensitive to channel variations.

Relaying of signals, as viewed from the physical layer, is not a trivial issue.  The authors in \cite{thesis:Laneman01}, \cite{article:Laneman01}, \cite{article:Laneman02} have provided several physical layer relay algorithms.  These include \emph{amplify-and-forward}, \emph{decode-and-forward} and \emph{selection relaying}.  In amplify-and-forward, a node amplifies its receive symbol, subject to a power constraint, before re-transmitting to the next node.  This algorithm is obviously with low complexity.  In decode-and-forward, a node fully decodes a symbol, re-encodes it and then re-transmits it.  In other words, this scheme attempts to eliminate channel distortion and noise at each node.  In selection relaying, a node only re-transmits a symbol if the measured receiving channel gain is above a certain threshold.  If the threshold is not reached, the relay requests a re-transmission from the sender.  In networking terminology, this is a type of automatic repeat request (ARQ) scheme.

The authors in \cite{article:Laneman01}, \cite{article:Laneman02} have investigated cooperation for the classical relay channel introduced in \cite{book:Cover01}, \cite{article:Laneman02}.  \emph{Outage probability} is used to characterize performance.  Outage probability is the probability that the mutual information between the transmitter and receiver does not reach a certain throughput threshold.  Without cooperation, the outage probability decays proportionally with $1/\mbox{SNR}$, where $\mbox{SNR}$ is the signal-to-noise ratio of the channel.  Using cooperation and the amplify-and-forward scheme, the outage probability decays proportionally with $1/\mbox{SNR}^2$, achieving \emph{full diversity}.  This results in large power savings for the transmitter.

The authors in \cite{article:Hasna02}, \cite{article:Hasna01} have investigated cooperation for a single path of relays connected in series.  The motivation for this network structure is that broader wireless coverage can be achieved, while still maintaining a low power constraint at the transmitter.  The authors consider \emph{analog relaying} and \emph{digital relaying} as two possible relay algorithms.  These are equivalent to the amplify-and-forward and decode-and-forward algorithms, respectively.  A power budget is considered where each packet travelling through the network is only allowed to consume a total fixed amount of power.  As well, each node has a certain transmit power limit.  The outage probability is then minimized by allocating power among the relay network under these power constraints.  This power allocation accounts for the channel conditions in the network in order to achieve the optimal outage probability.  Simulations indicate that $2$ dB of total power can be saved for 5 relays by using optimal power allocation instead of uniform power allocation.  This is for the decode-and-forward case.  However, at high $\mbox{SNR}$ values, the decode-and-forward case approximates the amplify-and-forward case.

The authors in \cite{article:Adve01} have investigated cooperation for multiple paths of relays connected in parallel.  In the conventional scheme, all relays participate using amplify-and-forward.  This is called \emph{all-participate amplify-and-forward} (AP-AF).  The authors also consider an algorithm where only one relay is selected in the transmission to maximize the mutual information.  This is called \emph{selection amplify-and-forward} (S-AF).  S-AF selects the relay which results in the maximum mutual information between transmitter and receiver.  Simulations of outage probability indicate that $5$ dB of SNR can be saved for 3 relays by using S-AF instead of AP-AF.  The authors in \cite{article:Ribeiro01} derive symbol error probabilities for multiple paths of relays.

\section{OFDM in Cooperative Communications}
In this thesis, we continue to investigate cooperation by applying OFDM signals to cooperative relay networks.  We consider a single path relay network and a multiple path relay network.  Using the amplify-and-forward relay algorithm, we derive the input-output relations and the mutual informations of both networks.  Using a power constraint at each relay, we consider two relay power allocation schemes.  The first is constant gain allocation, where the amplifying gain used in the amplify-and-forward algorithm is constant for all subcarriers.  The second is equal power allocation, where each subcarrier transmits the same power.  We simulate the mutual informations using these two relay power allocations.  Using the decode-and-forward relay algorithm, we derive input-output relations for both networks.  We simulate bit error rates (BERs) and word error rates (WERs) for the two networks using both the amplify-and-forward and decode-and-forward relay algorithms.

\section{Organization of Thesis}
The thesis is organized as follows.  In Chapter \ref{chap:sp}, we consider the single path relay network in \cite{article:Hasna02}, \cite{article:Hasna01}.  In Chapter \ref{chap:mp}, we consider a modified version of the multiple path relay network in \cite{article:Adve01} where the transmitter-receiver direct link is removed.  Notice that these latter two relay configurations are series and parallel analogs of each other.  As well, they do not involve a direct link between the transmitter and receiver.  Finally, Chapter \ref{chap:conclusion} concludes the thesis and provides future research directions.
