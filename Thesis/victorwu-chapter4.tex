\chapter{Conclusions}
\label{chap:conclusion}

\section{Contributions}
In this thesis, we investigate cooperation by applying OFDM signals to cooperative relay networks.  We consider the single path relay network and the multiple path relay network, as shown in Figures \ref{fig:sp_sm} and \ref{fig:mp_sm}, respectively.  

Using the amplify-and-forward relay algorithm, we derive the input-output relations of the equivalent end-to-end channel between the transmitter and receiver for both networks.  Next, we derive the mutual informations for the two networks as a function of the relay channels, relay power allocations, and the transmitter SNR.  Table \ref{tab:I} shows that the mutual informations share the same mathematical form on a system level.  They only differ in the $\mathbf{z}$ parameter, where $\mathbf{z} = \mathbf{z}_{\mbox{single}}$ and $\mathbf{z}_{\mbox{multiple}}$.  This indicates that although the relay networks are different, they are mathematically similar on a system level

Assuming that the transmitter and each relay has a transmit power of $P_{\mbox{tot}}$, we derive a transmit power constraint for for the $k^{\mbox{th}}$ OFDM subcarrier at each relay, for $k = 1$ to $N$.  Using the power constraint, we consider two relay power allocation schemes.  The first is constant gain allocation, where the amplifying gain used in the amplify-and-forward algorithm is constant for all subcarriers.  The second is equal power allocation, where each subcarrier transmits the same power.  The former scheme does not require CSI, while the latter one does.

We simulate the mutual informations using the two relay power allocation schemes.  Results indicate that equal power allocation gives a slightly higher mutual information for the single path relay network.  For the multiple path network, the mutual information is practically the same for both schemes.

Using the decode-and-forward relay algorithm, we derive the input-output relations for both networks.  The transmitter and each relay are assumed to have uniform power distributions in this case.

We simulate the error rate (BER and WER) performance for the two networks using both the amplify-and-forward and decode-and-forward relay algorithms.  For the single path relay network, amplify-and-forward gives very poor performance, because as we increase the distance between the transmitter and receiver (and thus, add more relays), more noise and channel distortion enter the system.  Decode-and-forward gives significantly better performance because noise and channel distortion are eliminated at each relay.  Nonetheless, decode-and-forward is of higher complexity than amplify-and-forward.  For the multiple path relay network, decode-and-forward again gives better performance than amplify-and-forward.  However, the performance gains are small compared to the single path relay network case.  This is because as we increase the number of paths between the transmitter and receiver (and thus, add more relays), the system inherently becomes more resistant to noise and channel distortion.  Therefore, decode-and-forward cannot provide any more significant improvement over amplify-and-forward.  As a result, the small performance gains provided by decode-and-forward may not necessarily justify the increased complexity over amplify-and-forward in this case.

\section{Future Research}
Future research includes investigating relay algorithms other than amplify-and-forward and decode-and-forward in OFDM-based cooperative networks.  For example, hybrid algorithms combining the two can be considered.  That is, depending on the channel conditions, a relay can choose different relay algorithms for different subcarrier symbols in an OFDM signal.  It can amplify-and-forward, decode-and-forward, or even just discard subcarrier symbols.  This in turn leads to more possibilities for relay power allocation.

In this thesis, we only investigate the single path relay network and the multiple path relay network.  Other general relay networks need to be considered in the context of OFDM as well.  This will lend more insight into developing a general theoretical framework for OFDM in cooperative relay networks.

\begin{table}
 \caption[Mutual Information Expressions]{Mutual Information Expressions for Amplify-and-Forward}
 \begin{center}
  \begin{tabular}{|l|c|}
\hline
& \\
  {Single Path Relay Network} &
$\frac{1}{N} \mathbf{e}_N^T \log_2 \left( \mathbf{e}_N + \mbox{SNR} \: \mathbf{z}_{\mbox{single}} \right)$ \\
& \\
\hline
& \\
  {Multiple Path Relay Network} & 
$\frac{1}{N} \mathbf{e}_N^T \log_2 \left( \mathbf{e}_N + \mbox{SNR} \: \mathbf{z}_{\mbox{multiple}} \right)$ \\
& \\
\hline
  \end{tabular}
  \label{tab:I}
 \end{center}
\end{table}